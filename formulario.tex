\documentclass[12pt,a4paper]{article}
\usepackage{amsmath}    % need for subequations
\usepackage{hyperref}   % use for hypertext links, including those to external documents and URLs
\usepackage{esint}
\usepackage{amsfonts}
\usepackage[utf8]{inputenc}
\usepackage{makeidx}
\usepackage{amsfonts}
\usepackage{fullpage}
\usepackage[spanish]{babel}


\title{Formulario de COTE - ETSIT UPM}
\author{Carlos García-Mauriño}
\date{\today}

\begin{document}

\maketitle

\twocolumn

\section{OPH}
\label{sec:ondas_planas_homogeneas}

Condición: $ \vec{E} \cdot \hat{n} = 0 $ y $ \vec{H} \cdot \hat{n} = 0 $.

\[ \lambda = \frac{2 \pi}{\beta}; \quad v_f = \frac{\omega_0}{\beta}  =
\frac{1}{\sqrt{\mu \varepsilon}} \]

\[ c = \frac{1}{\sqrt{\mu_0 \varepsilon_0}} \]

\[ \gamma_0 = \sqrt{-\omega^2 \mu \varepsilon} = \alpha + j \beta \]

\[ \vec{H_0} = \frac{\hat{n} \times \vec{E_0}}{\eta} \]

\[ \eta = \frac{j \omega \mu}{\gamma_0} = \sqrt{\frac{\mu}{\varepsilon}} \]

\subsection{Poynting}
\label{sub:poynting}

\subsubsection{Vector de Poynting instantáneo}
\label{ssub:vector_de_poynting_instantaneo}


\[ \vec{S} ( \vec{r}, t ) = \vec{E} ( \vec{r}, t ) \times \vec{H} ( \vec{r}, t
) \]

\subsubsection{Valor medio}
\label{ssub:valor_medio}

\[ \left< \vec{S} ( \vec{r}, t ) \right> \]
\[ = \frac{1}{2} \operatorname{Re}\left\{ \vec{E}( \vec{r}, \omega_0 ) \times
\vec{H^*} ( \vec{r}, \omega_0)  \right\} \]
\[ = \frac{1}{2} \vec{E_0} \vec{E_0^*} \operatorname{Re}\left( \frac{1}{\eta^*}
\right) e^{-2 \alpha z} \hat{z} \]
\[ = \frac{1}{2} \vec{H_0} \vec{H_0^*} \operatorname{Re}\left( \eta^* \right)
e^{-2 \alpha z} \hat{z} \]

\subsubsection{En el dominio de la frecuencia}
\label{ssub:en_el_dominio_de_la_frecuencia}

Parte real:

\[ \frac{1}{2} \operatorname{Re}\left\{ \int_V - \vec{J_{c}^*} \cdot
\vec{E_c} \cdot dV \right\} = \]
\[ = \frac{1}{2} \operatorname{Re}\left\{ \oint_S \left(\vec{E_{c}} \cdot
\vec{H_c^*} \right) \vec{dS} \right\} \] 
\[ + \frac{\omega_0 \varepsilon^{\prime\prime}}{2} \int_V | \vec{E_c} |^2 dV \]
\[ + \frac{\omega_0 \mu^{\prime\prime}}{2} \int_V | \vec{H_c} |^2 dV \]
\[ + \frac{\sigma}{2} \int_V | \vec{E_c} |^2 dV \]

Parte imaginaria:

\[ \frac{1}{2} \operatorname{Im}\left\{ \int_V - \vec{J_{c}^*} \cdot
\vec{E_c} \cdot dV \right\} = \]
\[ = \frac{1}{2} \operatorname{Im}\left\{ \oint_S \left(\vec{E_{c}} \cdot
\vec{H_c^*} \right) \vec{dS} \right\} \] 
\[ +2\omega_0 [<w_m>-<w_e>] \]

\subsection{Campo Máximo}
\label{sub:campo_maximo}

\[ \max{\left(|E_r|, |E_i|\right)} \]

\subsection{Polarización}
\label{sub:polarizacion}

\begin{itemize}
		\item \textbf{Lineal} si:
				\begin{itemize}
						\item $ E_{u1} $ ó $ E_{u2} \, = \, 0$
						\item $ \vec{E_{0r}} = \vec{E_{0i}} $ 
						
				\end{itemize}
		\item \textbf{Circular} si:
				\begin{itemize}
						\item $ E_{u1} = E_{u2} $ 
						\item $ |\vec{E_{0r}}| = |\vec{E_{0i}}| $ y $ \vec{E_{0r}} \perp \vec{E_{0i}} $ 
				\end{itemize}
		\item \textbf{Elíptica}:
				En el resto de situaciones.
\end{itemize}

\subsubsection{Relación axial}
\label{ssub:relacion_axial}

\[ r = \frac{E_{semieje mayor}}{E_{semieje menor}} \] 

\subsubsection{Sentido de polarización}
\label{ssub:sentido_de_polarizacion}

Mirando en la dirección de propagación, la polarización será:

\[ \hat{n} \cdot \left( \vec{E_r} \times \vec{E_i} \right) \geq 0
\rightarrow \mbox{negativa (a izquierdas)}. \]
\[ \hat{n} \cdot \left( \vec{E_r} \times \vec{E_i} \right) \leq 0
\rightarrow \mbox{positiva (a derechas)}. \]

\section{Relaciones constitutivas}
\label{sec:relaciones_constitutivas}

\[ \vec{D} (\vec{r}, t) = \]
\[ = \int_{V^{\prime}} \int_{t^{\prime}=-\infty}^{t}
\varepsilon (\vec{r}, \vec{r^{\prime\prime}}, t-t^{\prime})
\vec{E} (\vec{r}, t^{\prime}) dt^{\prime} dV^{\prime}
\, \left[\frac{C}{m^2} \right] \]

Si $ \varepsilon^{\prime\prime} > 0 $, la permitividad depende de la
frecuencia.

\section{Condiciones de contorno}
\label{sec:condiciones_de_contorno}

\subsection{Campo eléctrico}
\label{sub:campo_electrico}

Si $ \sigma = \infty $:

\[ \hat{n_{12}} \times ( \vec{E_2} - \vec{E_1} ) = 0 \]

\subsection{Campo magnético}
\label{sub:campo_magnetico}

Si $ \sigma = \infty $:

\[ \hat{n_{12}} \cdot ( \vec{B_2} - \vec{B_1} ) = 0 \]

\section{Incidencia de O.P. sobre obstáculos}
\label{sec:incidencia_de_o_p_sobre_obstaculos}

\[ w_{t} = w_{i} (1 - |\rho|) \]

\subsection{Coeficiente de reflexión}
\label{sub:coeficiente_de_reflexion}

\[ \rho (z) = \frac{\vec{E_{0 ref}} \cdot e^{+\gamma_1 z}}{\vec{E_{0 inc}}
\cdot e^{-\gamma_1 z}} = \rho (z=0^+) \cdot e^{+2\gamma_1 z} \]

\[ \rho_{12} = \frac{z_{12} - \eta_1}{z_{12} + \eta_1} \]

\[ E_{\max} = E_{i_{\max}} (1 + |\rho|) \]

\subsection{Impedancia}
\label{sub:impedancia}

\[ z_{12} = \eta_2
\frac{z_{23} \cos{(\beta d_2)} + j \eta_2 \sen{(\beta d_2)}}
{\eta_2 \cos{(\beta d_2)} + j z_{23} \sen{(\beta d_2)}} \]

\subsection{Coeficiente de onda estacionaria}
\label{sub:coeficiente_de_onda_estacionaria}

\[ R.O.E. = S = \frac{\left| \vec{E_{total}} \right|_{\max}}{\left|
\vec{E_{total}} \right|_{min}} \]
\[ = \frac{1+|\rho|}{1-|\rho|}  \]


\end{document}
