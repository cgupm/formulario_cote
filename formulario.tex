\documentclass[12pt,a4paper]{article}
\usepackage{amsmath}    % need for subequations
\usepackage{hyperref}   % use for hypertext links, including those to external documents and URLs
\usepackage{esint}
\usepackage{amsfonts}
\usepackage[utf8]{inputenc}
\usepackage{makeidx}
\usepackage{amsfonts}
\usepackage{fullpage}
\usepackage[spanish]{babel}


\title{Formulario de COTE - ETSIT UPM}
\author{Carlos García-Mauriño}
\date{\today}

\begin{document}

\maketitle

\twocolumn

\section{OPH}
\label{sec:ondas_planas_homogeneas}

\[ \lambda = \frac{2 \pi}{\beta}; \quad v_f = \frac{\omega_0}{\beta}  =
\frac{1}{\sqrt{\mu \varepsilon}} \]

\[ \gamma_0 = \sqrt{-\omega^2 \mu \varepsilon} = \alpha + j \beta \]

\[ \frac{\hat{n} \times \vec{E_0}}{\eta} \]

\[ \eta = \frac{j \omega \mu}{\gamma_0} = \sqrt{\frac{\mu}{\varepsilon}} \]

\subsection{Poynting}
\label{sub:poynting}

Vector de Poynting instantáneo:

\[ \vec{S} ( \vec{r}, t ) = \vec{E} ( \vec{r}, t ) \times \vec{H} ( \vec{r}, t
) \]

Valor medio:

\[ \left< \vec{S} ( \vec{r}, t ) \right> \]
\[ = \frac{1}{2} \operatorname{Re}\left\{ \vec{E}( \vec{r}, \omega_0 ) \times
\vec{H^*} ( \vec{r}, \omega_0)  \right\} \]
\[ = \frac{1}{2} \vec{E_0} \vec{E_0^*} \operatorname{Re}\left( \frac{1}{\eta^*}
\right) e^{-2 \alpha z} \hat{z} \]
\[ = \frac{1}{2} \vec{H_0} \vec{H_0^*} \operatorname{Re}\left( \eta^* \right)
e^{-2 \alpha z} \hat{z} \]

En el dominio de la frecuencia:

\[ \frac{1}{2} \operatorname{Re}\left\{ \int_V - \vec{J_{c}^*} \cdot
\vec{E_c} \cdot dV \right\} = \]
\[ = \frac{1}{2} \operatorname{Re}\left\{ \oint_S \left(\vec{E_{c}} \cdot
\vec{H_c^*} \right) \vec{dS} \right\} \] 
\[ + \frac{\omega_0 \varepsilon^{\prime\prime}}{2} \int_V | \vec{E_c} |^2 dV \]
\[ + \frac{\omega_0 \mu^{\prime\prime}}{2} \int_V | \vec{H_c} |^2 dV \]
\[ + \frac{\sigma}{2} \int_V | \vec{E_c} |^2 dV \]

\subsection{Campo Máximo}
\label{sub:campo_maximo}

\[ \max{\left(|E_r|, |E_i|\right)} \]

\subsection{Polarización}
\label{sub:polarizacion}

\begin{itemize}
		\item \textbf{Lineal} si:
				$ E_{u1} $ ó $ E_{u2} \, = \, 0$.
		\item \textbf{Circular} si:
				$ E_{u1} = E_{u2} $ 
		\item \textbf{Elíptica}:
				En el resto de situaciones.
\end{itemize}

\subsubsection{Relación axial}
\label{ssub:relacion_axial}

\[ r = \frac{E_{semieje mayor}}{E_{semieje menor}} \] 

\subsubsection{Sentido de polarización}
\label{ssub:sentido_de_polarizacion}

Mirando en la dirección de propagación, la polarización será:

\[ \hat{n} \cdot \left( \vec{E_r} \times \vec{E_i} \right) \geq 0
\rightarrow \mbox{negativa}. \]
\[ \hat{n} \cdot \left( \vec{E_r} \times \vec{E_i} \right) \leq 0
\rightarrow \mbox{positiva}. \]


\end{document}
